\chapter*{Conclusion}
\addcontentsline{toc}{chapter}{Conclusion}

Digital music synthesis and audio processing are ongoing fields of study. New synthesis methods --- such as physical modelling --- as well as incumbent ones are actively being researched and perfected, to provide ever crisper and higher-quality sound. Reverberation algorithms implemented in professional audio software have become many times more complex than the Schroeder Reverb presented in chapter 6, making digital sound resemble its natural counterpart ever more strikingly. Filters are continuously being improved as well, ensuring that frequency responses have the fastest possible roll-off in the transition band, highest amount of attenuation in the stop band and least in the pass band.\\

\noindent This thesis clearly did not attempt to examine such state-of-the-art techniques or cutting-edge research. An entire thesis would have to be dedicated to every chapter --- if not every section --- presented here, if it intended to examine what current methods yield the best possible results. Rather, what this thesis aimed at was describing the most straight-forward and well-established practices, that can be ameliorated and refined by the interested reader in the future. \\

\noindent However, one may argue that attempting to build a complete synthesis system that implements not the simplest techniques, but those that produce best results, is an undertaking too large for a single person. For this reason, the  synthesizer created for the purpose of this thesis, \emph{Anthem}, will be open-sourced and made available to the online community, for free and with no restrictions. The goal is to make \emph{Anthem} the go-to alternative to commercial music synthesizers, built by audio programming enthusiasts from all over the world, who will hopefully improve on the code base written for this thesis. The journey does not end here. It begins.

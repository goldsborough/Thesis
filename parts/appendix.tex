\begin{appendices}

  \chapter{Code Listings}

  \pagebreak

  \lstinputlisting[frame=single,caption={C++ implementation of a complete sine wave generator.},label={code:sine}]{code/sine.cpp}

  \pagebreak

  \lstinputlisting[frame=single,caption={C++ program to produce one period of an additively synthesized complex waveform, given a start and end iterator to a container of partials, a buffer length, a maximum, "master", amplitude, a boolean whether or not to apply sigma approximation and lastly a maximum bit width parameter.},label={code:add}]{code/add.cpp}

  \pagebreak

  \lstinputlisting[frame=single,caption={C++ implementation of single Envelope segments.},label={code:envseg}]{code/envseg.cpp}

  \pagebreak

  \lstinputlisting[frame=single,caption={Private members of the \texttt{ModDock} class.},label={code:moditem}]{code/moditem.cpp}

  \pagebreak

  \lstinputlisting[frame=single,caption={Definition of the \texttt{modulate} method in the \texttt{ModDock} class.},label={code:moddockmodulate}]{code/moddockmodulate.cpp}

  \pagebreak

  \lstinputlisting[frame=single,caption={Implementations of the various FM algorithms shown in Figure \ref{fig:fmalgs}. This code excerpt is from the \texttt{FM} class.},label={code:fm}]{code/fm.cpp}

  \pagebreak

  \lstinputlisting[frame=single,caption={\texttt{Wavefile} header file.},label={code:wavehpp}]{code/wave.hpp}

  \lstinputlisting[frame=single,caption={\texttt{Wavefile} implementation file.},label={code:wavecpp}]{code/wave.cpp}


\end{appendices}
